\documentclass{resume} % Use the custom resume.cls style
\usepackage{enumitem}
\usepackage[left=0.4 in,top=0.4in,right=0.4 in,bottom=0.4in]{geometry} % Document margins
\newcommand{\tab}[1]{\hspace{.2667\textwidth}\rlap{#1}} 
\newcommand{\itab}[1]{\hspace{0em}\rlap{#1}}


\name{Albin P James} % Your name
\address{Thiruvananthapuram, Kerala, India \\ 9447029214 \\ albinpjames@outlook.com} % Your secondary addess (optional)
\address{https://github.com/albinpjames/ \\ https://linkedin.com/in/albinpjames/}  % Your phone number and email

\begin{document}

\begin{rSection}{Education}
{\bf Master's in Physics }, National Institute of Technology, Calicut (NITC) \hfill {10/2020 - 05/2022} \\
{\bf Bachelor’s in Physics}, St. Berchmans’ College, Kerala \hfill {06/2017 - 04/2020}\\
{\bf Senior Secondary}, Navajeevan Bethany Vidayalay CBSE, Nalanchira, Kerala \hfill {05/2017}
\end{rSection}

\begin{rSection}{ACADEMIC ACHIEVEMENTS}
\textbf{Joint Entrance Screening Test (JEST) Physics}  \hfill 2020 
\\ All India Rank \textbf{253}
\\A GRE-type national level entrance examination for postgraduate and Integrated PhD studies
in India conducted jointly by several premier institutes such as IISc, IISERs, NISER and other research institutes.
\\ \textbf{Joint Admission test for M.Sc (JAM) Physics }  \hfill 2020
\\ All India Rank \textbf{1116}
\\A GRE-type national level entrance examination for postgraduate and Integrated PhD studies
in India conducted by a group of Indian Institute of Technologies (IITs)
and other research institutes.

\end{rSection}

\begin{rSection}{AREAS OF INTEREST }
Computational Astrophysics and Cosmology \textbar{} \iffalse Computational Physics  \textbar{} \fi  Scientific Computing \textbar{}  History \& Philosohy of Science
\end{rSection}

\begin{rSection}{RESEARCH \& PROJECT EXPERIENCE}


\textbf {Statistical Properties of Dark Matter Halos from Cosmological Simulations}\hfill 07/2021 - 05/2022 \\ 
MSc Dissertation \\
Supervisor: Dr Charles Jose, Assistant Professor, CUSAT \\
In the theory of cosmological structure formation, galaxies form in gravitationally
bound dark matter clumps called dark matter halos. In this project, we investigate
the statistical properties of dark matter halos using numerical simulations. In
particular, we estimate the halo mass function, which is the number density of dark
matter halo as a function of mass, for various redshifts and box sizes. Further, a
counts-in-cells analysis for halo mass function was conducted for redshifts 0.2 and
3 by dividing the boxes into smaller boxes of 50, 25 and 20 MPc/h and measuring
the variation of mass function across sub boxes\\


\textbf {Development of MATLAB code for the analysis of Magnetometer data}  \hfill 07/2019 - 05/2020 \\ 
BSc Dissertation\\
Supervisor: Dr Lijo Jose, Assistant Professor, SB College \\ 
The data from Sensys FGM3D TD magnetometer was analysed using MATLAB codes and the output processed and plotted using Origin Pro. Magnetometers of Space Physics Laboratory, VSSC Trivandrum placed at Changanassery SB College and Kannur were used to collect the data. The analysis of the magnetometer data during the solar eclipse using the code showed a drastic variation of the ionosphere compared to an average day. 
\end{rSection} 

\begin{rSection}{Professional Certification}

\textbf{Google IT Support Professional Certification} \hfill 07/2020\\
Google and offered through Coursera. The courses taken includes
\begin{enumerate}
	\itemsep-0.25cm
	\item Technical Support Fundamentals
	\item Operating Systems and You: Becoming a Power User
	\item IT Security: Defense against the digital dark arts
	\item System Administration and IT Infrastructure Services
	\item The Bits and Bytes of Computer Networking
\end{enumerate}

\end{rSection}

\begin{rSection}{TECHNICAL SKILLS \& EXPERTISE}{}
	\textbf {Programming Languages:}\\ 
	Familiar with: Python, C++, MATLAB, LaTeX \textbar{} Basic understanding of: C, HTML, CSS  
	
	\item \textbf {Miscellaneous:} Origin Pro, Slrum \\
	\textbf { Python Packages \& Simulations :} sklearn, abacusnbody, astropy, scipy, numpy, GADGET2, NgenIC, etc... \\
	\textbf { Platforms:} Linux, Windows, MS Office \\ 
	\textbf { Visual \& Graphics:} After Effects, Premier Pro, Photoshop, Illustrator 
\end{rSection}

\begin{rSection}{Certifications}
\textbf{Machine Learning with Python }\hfill Ongoing\\
IBM and offered through Coursera \\
\textbf{Data-driven Astronomy }\hfill Ongoing\\
The University of Sydney and offered through Coursera \\
\textbf{From the Big Bang to Dark Energy }\hfill 06/2020\\
The University of Tokyo and offered through Coursera \\
\textbf{Understanding Research Methods }\hfill 04/2020\\
University of London and SOAS University of London and offered through Coursera  \\
\textbf{Psychological First Aid} \hfill 12/2019\\
Johns Hopkins University and offered through Coursera \\
\textbf{Philosophy and the Sciences: Introduction to the Philosophy of Physical Sciences } \hfill 08/2019\\
The University of Edinburgh and offered through Coursera  \\
\textbf{Introduction to Psychology }\hfill 07/2019\\
Yale University and offered through Coursera 
\end{rSection}


\begin{rSection}{MEMBERSHIPS}
\textbf{Breakthrough Science Society, Trivandrum Chapter} \hfill 5/2021
\\ A voluntary organisation committed to the cause of science, culture and scientific outlook.
\begin{itemize}[noitemsep,topsep=-0.2cm]
	\itemsep-0.05cm
	\item To promote science among students \& general public.
	\item To provide students access to seminars and talks by notable researchers free of cost.
\end{itemize}
\end{rSection}


\begin{rSection}{CONFERENCES AND WORKSHOPS }
	\textbf{Sagan Exoplanet Summer Workshop on "Circumstellar Disks and Young Planets"} \hfill 2021\\
	NASA Exoplanet Science Institute, California Institute of Technology in Pasadena, CA.  \\
	\textbf{IIA Summer Programme 2021}\hfill 2021\\
	Indian Institute of Astrophysics  \\
	\textbf{E-Workshop on LaTeX 2021 }\hfill 2021\\
	GITAM Deemed to be University \\
	\textbf{Creative Science Workshop 2012 } \hfill 2012\\
	Kerala State Science and Technology Museum (KSSTM), Trivandrum \\
\end{rSection}

\begin{rSection}{POSITIONS OF RESPONSIBILITIES}
	
	\textbf{Deputy Speaker } \hfill 2021-2022\\
	Student Affairs Council (SAC) - NIT Calicut,
	\begin{itemize}[noitemsep,topsep=-0.2cm]
		\itemsep-0.25cm
		\item I was entrusted with coordinating all the activities pertaining to student body including the research community. 
	\end{itemize}
	
	\textbf{Student Coordinator  } \hfill 2018-2020\\
	Science Centre- St Berchmans College, Changanassery,
	\begin{itemize}[noitemsep,topsep=-0.2cm]
		\itemsep-0.05cm
		\item I researched and hosted an Inter departmental science quiz. 
		\item Organised an Astronomy workshop and various sky watching sessions. 
		\item Hosted science exhibitions as part of the science centre. 
	\end{itemize}
	
	\textbf{Secretary} \hfill 2019-2020\\
	Physics Department Association- St Berchmans College, Changanasserry 
	\begin{itemize}[noitemsep,topsep=-0.2cm]
		\itemsep-0.05cm
		\item I was instrumental in initiating several programmes and activities in the college on behalf of the association. 
		\item Organized an All Kerala Inter Collegiate Quiz and presentation competition. 
	\end{itemize}

	
	\textbf{HRD Team Excellence Member } \hfill 2018-2019\\
	Human Resource Department- St Berchmans College, Changanasserry \\
	Every programmes in the college are carried out by a group of 20 students selected by the Human Resource Department of the college known as 'Team Excellence'. 
	\begin{itemize}[noitemsep,topsep=-0.2cm]
		\itemsep-0.05cm
		\item I was able to co-ordinate with my team to organize several programmes in the college. 
		\item Organized several National and International seminars in the college. 
	\end{itemize}
\end{rSection}

\begin{rSection}{ACHIEVEMENTS }
\textbf{MG University Kalolsavam Quiz}\\
Represented St Berchmans College at MG University Kalolsavam and bagged the runner up position for the consecutive years 2019 and 2020. \\
\textbf{Winner - Major Science  Quiz Competitions }\\
Won prizes for all major quiz competitions conducted across Kerala. Mainly the one focusing on science. Space and physics quiz are my forte.\\
\textbf{Best Outgoing Student}\\
The best outgoing student of Navajeevan Bethany Vidyalaya during the academic year 2016-2017.\\
\end{rSection} 



\begin{rSection}{Courses}
	Numerical \& Computational Physics \textbar{} Astrophyics \textbar{} General Relativity \textbar{} Classical Mechanics \textbar{} Electromagnetic Theory \textbar{} Mathematical Physics \textbar{} Nuclear and particle Physics \textbar{}
	Quantum Mechanics \textbar{} Condensed Matter Physics.
\end{rSection}


\end{document}
